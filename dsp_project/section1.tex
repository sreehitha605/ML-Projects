\section{Introduction of Crop Recommendation System}
      In India,farming is not considered as a business but also has a huge impact on the social life of people which are associated to it. There are many festivals and social gatherings are celebrated in accordance with the different seasons and practices involved in farming. Hence large part of population is dependent on agriculture field directly or indirectly.Most of the times it is observed that farmers tend to saw the crop according to its market value and possible financial profits rather than taking factors like soil conditions,weather conditions etc., into the account.This may lead to undesirable results for farmers and for the nature of soil too.\par
      In today's time,technologies like machine learning and deep learning can become the game changers in such fields if they are used in proper manner.This project aim is to recommend the most suitable crop based on input parameters like Nitrogen(N),Phosphorus(P),
Potassium(K),pH value of the soil,Humidity,Temperature and Rainfall.This project predicts the accuracy of the future prodution of nine different crops such as pomegranate,banana,
mango,grapes,watermelon,muskmelon,apple,orange,papaya crops using various supervised machine learning approches and recommends the most suitable crop.\par This dataset contains various parameters like Nitrogen(N),Phosphorus(P),Potassium(K),
PH value of the soil,Humidity,Temperature and Rainfall. This proposed system applied different kinds of machine learning algorithms like K-Nearest Neighbours(K-NN),Decision tree,Support Vector Machine(SVM) to predict the suitable crop.
\section{Application}
     A crop recommedation system is a valuable application that can help farmers in making better decisions about which crops to grow based on various factors such as soil quality and climate conditions. It can suggest suitable crop for a specific region or field based on the parameters it helps farmers select crops that have higher chances of success and profitability. By recommending crops that are well suited to the conditions farmers can maximize crop productivity. It leads to better yeilds.It helps farmers optimize the use of fetilizers,water etc.,and reducing wastage.
\section{Motivation Towards Crop Recommendation System}
     Agriculture plays a crucial role in Indian economy and employment. Half of the country's population is still employed on agriculture sector. India is one of the largest producers of agricultural products, but still it has less farm production. The most common problem faced by the farmers is that they do not opt the crop based on the necessity of soil. This problem can be solved by using crop recommendation system.\par
     A crop recommendation system project aim is to recommend the most suitable crop based on input parameters like Nitrogen(N),Phosphorus(P),
Potassium(K),pH value of the soil,Humidity,Temperature and Rainfall.This project predicts the accuracy of the future prodution of nine different crops such as pomegranate,banana,mango,grapes,watermelon,
muskmelon,apple,orange,papaya crops using various supervised machine learning approches and recommends the most suitable crop.
\section{Problem Statement}
     There are very few platforms that helps farmers with their farming strategy and crop recommedation is one of them. Farmer understimate the fertility of the soil on their farms.The most common problem faced by the farmers is that they do not opt the crop based on the necessity of soil. This problem can be solved by using crop recommendation system.Using appropriate parameters like rainfall,temperature,humidity,nutrient levels etc.This system can predict the suitable crop for a specific field.











